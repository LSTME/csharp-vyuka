\chapter{Prostredie a premenné}

Každý účastník si spustí Unity a vytvorí nový projekt. Projekt pomenujte \textbf{Projekt1}.

Vyrvorte adresár \code{Assets/Scenes} a uložte doňho súčasnú nepomenovanú scénu ako \code{Main.unity}. Pozn. podľa operačného systému sa automaticky každému zvolí najvhodnejšie grafické API, môžu s tým nastať mierne komplikácie, preto na to pamätať alebo každému odporučiť zmeniť grafické API na OpenGL Core. (Edit \textgreater Project Settings \textgreater Player \textgreater Other Settings \textgreater Rendering, vypnúť automatické pridelenie grafického API a zvoliť so zoznamu, engine vyberá zhora nadol prvé, ktoré môže použiť.)

Ukážte účastníkom, čo je kamera, jej nastavenia. Zmente \code{Clear Flags} na  \code{Solid Color} a \code{Background} na \code{314D79FF}. Môžete spomenúť rozdieľ medzi perspektívou a ortografickým premietaním (\code{Projection} nastavenie), to môže byť však ukázané neskôr. Tak isto je vhodné teraz ukázať, ako sa pohybuje s objektom (klávesa W), rotuje objekt (klávesa E) a škáluje objekt (klávesa R).

V tomto cvičení sa budeme zaoberať iba GUI. Vložte do scény UI element Text. Oddialte kameru scény a účastníkom ukážte čo je Canvas a kde vidia Text objekt. Zmente polohu textu na 0, 0, 0 a nastavte ukotvenie na stred. Ukážte účastníkom ako sa spustí hra, mali by vidieť čierny text na modrom pozadí v strede obrazovky.

Označte text a zmeňte mu rozmery na 640 x 60. Zmente hodnotu textu napríklad na ,,Ahoj hráč!'' a zmente veľkosť fontu tak, aby bolo text dobre vidieť. Napríklad 36 bodov. Môžete ukázať zarovnanie textu (nakoniec ho zarovnajte na stred horizontálne aj vertikálne). Zmente farbu na \code{FFFFFFFF}.

Ideme programovať! Všetci si vytvoria adresár \code{Assets/Scripts} a v ňom vytvoria C\# skript \code{MojText}. Pridajte komponent s týmto skriptom do existujúceho Textového UI elementu. Otvorte skript a vysvetlite, kedy sa volá \code{Start} a kedy \code{Update}. Do \code{Start} napíšeme jeden riadok, ktorý zmení text na ,,Ahoj -meno účastníka-'' \refCode{lst:MojText1}.

\clearpage
\lstinputlisting[language=CSharp,caption=Prvý skript.,label=lst:MojText1]{./codes/premenne_a_prostredie/MojText_1.cs}

Lepšie by bolo ale mať meno ako parameter v skripte, preto prídáme prvú premennú \code{public string meno;} do skriptu a ukážeme účastníkom, že sa teraz nachádza v komponente skriptu v objekte Text. Následne \code{meno} použijeme v texte \refCode{lst:MojText2}.

\lstinputlisting[language=CSharp,caption=Prvá premenná.,label=lst:MojText2]{./codes/premenne_a_prostredie/MojText_2.cs}

Ukážeme si ešte \code{float} a \code{int}. Pridáme \code{public float vyska;} a \code{public int vek;}. Potom urpavíme vypisovaný text s vekom a výškou \refCode{lst:MojText3}.

\clearpage
\lstinputlisting[language=CSharp,caption=Viacero premenných.,label=lst:MojText3]{./codes/premenne_a_prostredie/MojText_3.cs}